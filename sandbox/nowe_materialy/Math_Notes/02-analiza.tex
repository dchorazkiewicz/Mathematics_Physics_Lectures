\chapter{Analiza Matematyczna}

Poniższy rozdział jest zorganizowany modułowo. Skupiamy się tutaj na ścisłym, formalnym wyprowadzeniu dowodów oraz jawnym przeliczeniu przykładów krok po kroku.

% --- PLAN ROZDZIAŁU ---
% 1. Definicja granicy ciągu (epsilon-N) + intuicja + proste dowody z definicji.
% 2. Arytmetyka granic (twierdzenia o sumie, iloczynie) - dowody i przykłady.
% 3. Twierdzenie o trzech ciągach.
% 4. Liczba e i ciągi z nią związane.

% ==============================================================================
% SEKCJA 1: DEFINICJA GRANICY
% ==============================================================================
\section{Definicja granicy ciągu}
W tej sekcji wprowadzamy formalizm Cauchy'ego i pokazujemy, jak go używać w praktyce do weryfikacji zbieżności prostych ciągów.

% % Moduł: Formalna definicja granicy ciągu i przykłady

\subsection*{Formalna definicja granicy ciągu}

Mówimy, że ciąg \( (a_n) \) ma granicę \( g \in \mathbb{R} \), jeśli dla każdej dowolnie małej liczby \(\varepsilon > 0\) istnieje taka liczba naturalna \( N \), że dla wszystkich \( n > N \) spełniona jest nierówność:
\[
|a_n - g| < \varepsilon
\]
Piszemy wówczas \(\displaystyle \lim_{n\to\infty} a_n = g\).

\paragraph{Idea definicji}
Definicja mówi, że bez względu na to, jak wąski pasek o szerokości \(2\varepsilon\) narysujemy wokół wartości granicznej \(g\), od pewnego miejsca \(N\) wszystkie kolejne wyrazy ciągu \(a_n\) muszą się w tym pasku znaleźć. Liczba \(N\) zależy od wybranego \(\varepsilon\).

\paragraph{Przykład 1: Dowód, że \(\displaystyle \lim_{n\to\infty} \frac{1}{n} = 0\)}
Chcemy pokazać, że dla dowolnego \(\varepsilon > 0\) potrafimy znaleźć takie \(N\), że dla każdego \(n > N\) zachodzi |\(\frac{1}{n}\) - 0| < \(\varepsilon\).

Krok 1: Rozwiązujemy nierówność ze względu na \(n\).
\[
\left|\frac{1}{n}\right| < \varepsilon
\]
Poniważ \(n\) jest liczbą naturalną, więc \(n>0\), zatem |\(\frac{1}{n}\)| = \(\frac{1}{n}\).

\[
\frac{1}{n} < \varepsilon \quad \iff \quad n > \frac{1}{\varepsilon}
\]
Krok 2: Wybieramy \(N\).
Nierówność \(n > 1/\varepsilon\) będzie spełniona dla wszystkich \(n\) większych od \(1/\varepsilon\). Możemy więc wybrać \(N\) jako dowolną liczbę naturalną większą lub równą \(1/\varepsilon\). Formalnie, niech \(N = \lfloor \frac{1}{\varepsilon} \rfloor + 1\).

Krok 3: Formalne sprawdzenie.
Niech \(\varepsilon > 0\) będzie dowolne. Wybierzmy \(N = \lfloor \frac{1}{\varepsilon} \rfloor + 1\). Wtedy dla każdego \(n > N\) mamy:
\[
n > \lfloor \frac{1}{\varepsilon} \rfloor + 1 \ge \frac{1}{\varepsilon}
\]
A zatem \(n > 1/\varepsilon\), co po przekształceniu daje \(1/n < \varepsilon\). Ponieważ \(1/n > 0\), jest to równoważne |\(\frac{1}{n}\) - 0| < \(\varepsilon\). To kończy dowód.

\paragraph{Przykład 2: Dowód, że \(\displaystyle \lim_{n\to\infty} \frac{n^2-1}{n^2} = 1\)}
Chcemy pokazać, że dla dowolnego \(\varepsilon > 0\) istnieje takie \(N\), że dla każdego \(n > N\) zachodzi |\(\frac{n^2-1}{n^2}\) - 1| < \(\varepsilon\).

Krok 1: Upraszczamy wyrażenie i rozwiązujemy nierówność.
\[
\left|\frac{n^2-1}{n^2} - 1\right| = \left|\frac{n^2-1 - n^2}{n^2}\right| = \left|\frac{-1}{n^2}\right| = \frac{1}{n^2}
\]
Teraz rozwiązujemy nierówność:
\[
\frac{1}{n^2} < \varepsilon \quad \iff \quad n^2 > \frac{1}{\varepsilon} \quad \iff \quad n > \frac{1}{\sqrt{\varepsilon}}
\]
(Bierzemy tylko dodatni pierwiastek, bo \(n>0\)).

Krok 2: Wybieramy \(N\).
Musimy znaleźć \(N\) takie, że dla \(n>N\) spełnione jest \(n > 1/\sqrt{\varepsilon}\). Możemy wybrać \(N = \lfloor \frac{1}{\sqrt{\varepsilon}} \rfloor + 1\).

Krok 3: Formalne sprawdzenie.
Niech \(\varepsilon > 0\) będzie dowolne. Wybierzmy \(N = \lfloor \frac{1}{\sqrt{\varepsilon}} \rfloor + 1\). Wtedy dla każdego \(n > N\) mamy:
\[
n > \frac{1}{\sqrt{\varepsilon}}
\]
Przekształcając, otrzymujemy kolejno \(n^2 > 1/\varepsilon\), a następnie \(1/n^2 < \varepsilon\). To jest równoważne pokazanej wcześniej nierówności |\(\frac{n^2-1}{n^2}\) - 1| < \(\varepsilon\). To kończy dowód. % UWAGA: Plik nie istnieje!

% ==============================================================================
% SEKCJA 2: OBLICZENIA JAWNE I ARYTMETYKA GRANIC
% ==============================================================================
\section{Obliczenia jawne granic}
Tutaj zajmiemy się dowodzeniem zbieżności bardziej złożonych wyrażeń (wielomiany, funkcje trygonometryczne), wykorzystując arytmetykę granic.

% Moduł o granicach funkcji wymiernych
% Moduł: Granice funkcji wymiernych — technika wyciągania najwyższej potęgi

\subsection*{Granice funkcji wymiernych: wyciąganie najwyższej potęgi}

Celem jest obliczanie granic funkcji wymiernych (ilorazów wielomianów) przez jawne wyciągnięcie najwyższej potęgi zmiennej przed nawias. Metoda jest przejrzysta, stabilna obliczeniowo i prowadzi do poprawnego wyniku zarówno dla $x\to \infty$, $x\to -\infty$, jak i dla punktów skończonych, w których licznik i mianownik jednocześnie dążą do zera.

\paragraph{Idea metody}
Niech \(P(x)\) i \(Q(x)\) będą wielomianami stopnia odpowiednio \(n\) i \(m\):
\[
P(x) = a_n x^n + a_{n-1} x^{n-1} + \dots + a_0,\quad\quad Q(x) = b_m x^m + b_{m-1} x^{m-1} + \dots + b_0,
\]
przy czym \(a_n\neq 0\) i \(b_m\neq 0\).

Rozważamy granicę \(\displaystyle \lim_{x\to\infty} \frac{P(x)}{Q(x)}\). Kluczowy krok:
- wyciągamy z licznika \(x^n\), a z mianownika \(x^m\),
- następnie upraszczamy iloraz \(x^n/x^m\) oraz analizujemy pozostałe czynniki, które stają się współczynnikami (lub znikają) w granicy.

\paragraph{Formalny zapis kroku wyciągania}
Dla \(x\neq 0\):
\begin{align}
\frac{P(x)}{Q(x)}
&= \frac{x^n\left(a_n + a_{n-1}\,x^{-1} + \dots + a_0\,x^{-n}\right)}{x^m\left(b_m + b_{m-1}\,x^{-1} + \dots + b_0\,x^{-m}\right)} \\
&= x^{n-m} \cdot \frac{a_n + a_{n-1}\,x^{-1} + \dots + a_0\,x^{-n}}{b_m + b_{m-1}\,x^{-1} + \dots + b_0\,x^{-m}}.
\end{align}
Ponieważ \(\lim_{x\to\infty} x^{-k} = 0\) dla każdego \(k\in\mathbb{N}\), dostajemy trzy przypadki:
\[
\lim_{x\to\infty} \frac{P(x)}{Q(x)} =
\begin{cases}
0, & n < m, \\
\dfrac{a_n}{b_m}, & n = m, \\
\operatorname{sgn}(x^{n-m})\cdot \infty & n > m\ \text{(tj. granica niewłaściwa; rośnie/jest nieograniczona)}.
\end{cases}
\]
W przypadku \(x\to -\infty\) znak \(x^{n-m}\) zależy od parzystości \(n-m\).

\paragraph{Przykład 1: wielomian przez wielomian dla $x\to\infty$}
Obliczmy \(\displaystyle \lim_{x\to\infty} \frac{2x^3 - 5x^2 + x - 7}{x^3 + 4x - 1}\).

Wyciągamy najwyższą potęgę \(x^3\) z licznika i mianownika:
\begin{align}
\frac{2x^3 - 5x^2 + x - 7}{x^3 + 4x - 1}
&= \frac{x^3\left(2 - 5\,x^{-1} + 1\,x^{-2} - 7\,x^{-3}\right)}{x^3\left(1 + 4\,x^{-2} - 1\,x^{-3}\right)} \\
&= \frac{2 - 5\,x^{-1} + x^{-2} - 7\,x^{-3}}{1 + 4\,x^{-2} - x^{-3}}.
\end{align}
Każdy składnik \(x^{-k}\) dąży do zera, stąd
\[
\lim_{x\to\infty} \frac{2x^3 - 5x^2 + x - 7}{x^3 + 4x - 1} = \frac{2}{1} = 2.
\]
To jest zgodne z zasadą: stopnie równe \(n=m=3\) dają iloraz wiodących współczynników \(a_n/b_m\).

\paragraph{Przykład 2: przypadek $n<m$}
Obliczmy \(\displaystyle \lim_{x\to\infty} \frac{3x^2 - x + 4}{5x^4 + x}\).

Wyciągamy najwyższe potęgi:
\begin{align}
\frac{3x^2 - x + 4}{5x^4 + x}
&= \frac{x^2\left(3 - 1\,x^{-1} + 4\,x^{-2}\right)}{x^4\left(5 + 1\,x^{-3}\right)} \\
&= x^{-2}\cdot \frac{3 - x^{-1} + 4\,x^{-2}}{5 + x^{-3}}.
\end{align}
Ponieważ \(x^{-2} \to 0\) i pozostały ułamek dąży do \(\frac{3}{5}\), całość dąży do zera:
\[
\lim_{x\to\infty} \frac{3x^2 - x + 4}{5x^4 + x} = 0.
\]

\paragraph{Przykład 3: przypadek $n>m$}
Obliczmy \(\displaystyle \lim_{x\to\infty} \frac{x^5 - 2x}{3x^2 + 1}\).
\begin{align}
\frac{x^5 - 2x}{3x^2 + 1}
&= \frac{x^5\left(1 - 2\,x^{-4}\right)}{x^2\left(3 + 1\,x^{-2}\right)} \\
&= x^{3}\cdot \frac{1 - 2\,x^{-4}}{3 + x^{-2}}.
\end{align}
Iloraz w nawiasie dąży do \(\frac{1}{3}\), ale czynnik \(x^3\to \infty\), zatem granica jest niewłaściwa (również \(+\infty\) dla \(x\to\infty\)).

\paragraph{Granice w punktach skończonych z nieoznaczonością $\frac{0}{0}$}
Rozważmy \(\displaystyle \lim_{x\to 1} \frac{x^3 - 1}{x^2 - 1}\). W punkcie \(x=1\) zarówno licznik, jak i mianownik są równe zero. Zastosujemy jawny rozkład na czynniki oraz wyciągnięcie potęg względem \((x-1)\).

Krok 1: rozkład wielomianów.
\[
x^3 - 1 = (x-1)(x^2 + x + 1),\quad x^2 - 1 = (x-1)(x+1).
\]
Krok 2: skrócenie wspólnego czynnika \((x-1)\) dla \(x\neq 1\):
\begin{align}
\frac{x^3 - 1}{x^2 - 1}
&= \frac{(x-1)(x^2 + x + 1)}{(x-1)(x+1)} \\
&= \frac{x^2 + x + 1}{x+1}.
\end{align}
Krok 3: obliczenie granicy poprzez ciągłość: otrzymana funkcja jest ciągła w \(x=1\), więc
\[
\lim_{x\to 1} \frac{x^3 - 1}{x^2 - 1} = \left.\frac{x^2 + x + 1}{x+1}\right|_{x=1} = \frac{1 + 1 + 1}{2} = \frac{3}{2}.
\]
Alternatywnie, można potraktować oba wielomiany jako szeregi w zmiennej \(h = x-1\), wyciągając najwyższą wspólną potęgę \(h\).

\paragraph{Uwagi o $x\to -\infty$}
W przypadku granic dla \(x\to -\infty\) należy dodatkowo śledzić znak \(x^{n-m}\). Jeżeli \(n=m\), wynik pozostaje \(a_n/b_m\). Jeżeli \(n>m\), to:
\[
\lim_{x\to -\infty} x^{n-m} =
\begin{cases}
+\infty, & n-m \text{ parzyste}, \\
-\infty, & n-m \text{ nieparzyste}.
\end{cases}
\]
Iloraz w nawiasie nadal dąży do \(a_n/b_m\), więc kierunek nieograniczonego wzrostu zależy od parzystości.

\paragraph{Podsumowanie}
Metoda wyciągania najwyższej potęgi sprowadza granicę funkcji wymiernej do analizy prostych czynników \(x^{n-m}\) oraz stabilnego ilorazu współczynników wiodących. Dla punktów skończonych z nieoznaczonością \(\frac{0}{0}\) najlepiej jawnie wyciągać wspólny czynnik \((x-c)\) lub korzystać z rozkładu na czynniki, co zapewnia przejrzyste i formalnie poprawne obliczenie granicy.


% \input{analiza_02_obliczenia} % Tutaj agenci będą wpinać kolejny plik