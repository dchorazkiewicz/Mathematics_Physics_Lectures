\section*{Limits}

\subsection*{Limit of a Sequence}

\paragraph{Formal Definition}
A number $L \in \mathbb{R}$ is the limit of a sequence of real numbers $(a_n)_{n=1}^{\infty}$ if for every real number $\varepsilon > 0$, there exists a natural number $N$ such that for all $n > N$, the inequality $|a_n - L| < \varepsilon$ holds.

We write this as:
\[ \lim_{n \to \infty} a_n = L \]
which is equivalent to:
\[
\forall_{\varepsilon > 0} \quad \exists_{N \in \mathbb{N}} \quad \forall_{n > N} \quad |a_n - L| < \varepsilon
\]

\paragraph{Example: Proof of convergence for the sequence $a_n = 1/n$}
We will prove that the limit of the sequence $a_n = 1/n$ is 0.
\[ \lim_{n \to \infty} \frac{1}{n} = 0 \]
\textbf{Proof:}
According to the formal definition of a limit, we need to show that for any given $\varepsilon > 0$, there exists a natural number $N$ such that for all $n > N$, the inequality $|a_n - L| < \varepsilon$ is satisfied.

In our case, $a_n = 1/n$ and $L = 0$. The inequality becomes:
\[ \left|\frac{1}{n} - 0\right| < \varepsilon \]
Since $n$ is a natural number, $n > 0$, so $|1/n| = 1/n$. The inequality simplifies to:
\[ \frac{1}{n} < \varepsilon \]
To find an appropriate $N$, we can solve this inequality for $n$:
\[ n > \frac{1}{\varepsilon} \]
By the Archimedean property of real numbers, for any positive real number $\varepsilon$, there exists a natural number $N$ such that $N > 1/\varepsilon$. Let's choose $N = \lceil 1/\varepsilon \rceil$.

Now, for any $n > N$, we have:
\[ n > N \ge \frac{1}{\varepsilon} \]
This implies that $1/n < \varepsilon$. Therefore, we have shown that for any $\varepsilon > 0$, we can find an $N$ (for example, $N = \lceil 1/\varepsilon \rceil$) such that for all $n > N$, $|1/n - 0| < \varepsilon$. This completes the proof that the limit of the sequence $1/n$ is 0.
\hfill $\blacksquare$

\paragraph{Example: Proof of convergence for the sequence $a_n = \frac{n^2-1}{n^2}$}
We will prove that the limit of the sequence $a_n = \frac{n^2-1}{n^2}$ is 1.
\[ \lim_{n \to \infty} \frac{n^2-1}{n^2} = 1 \]
\textbf{Proof:}
According to the formal definition of a limit, we need to show that for any given $\varepsilon > 0$, there exists a natural number $N$ such that for all $n > N$, the inequality $|a_n - L| < \varepsilon$ is satisfied.

In our case, $a_n = \frac{n^2-1}{n^2}$ and $L = 1$. The inequality becomes:
\[ \left|\frac{n^2-1}{n^2} - 1\right| < \varepsilon \]
Let's simplify the expression inside the absolute value:
\[ \left|\frac{n^2-1 - n^2}{n^2}\right| = \left|-\frac{1}{n^2}\right| = \frac{1}{n^2} \]
The inequality simplifies to:
\[ \frac{1}{n^2} < \varepsilon \]
To find an appropriate $N$, we can solve this inequality for $n$:
\[ n^2 > \frac{1}{\varepsilon} \]
\[ n > \frac{1}{\sqrt{\varepsilon}} \]
Let's choose $N = \lceil 1/\sqrt{\varepsilon} \rceil$.

Now, for any $n > N$, we have:
\[ n > N \ge \frac{1}{\sqrt{\varepsilon}} \]
This implies that $n^2 > 1/\varepsilon$, and therefore $1/n^2 < \varepsilon$. Thus, we have shown that for any $\varepsilon > 0$, we can find an $N$ such that for all $n > N$, $|\frac{n^2-1}{n^2} - 1| < \varepsilon$. This completes the proof that the limit of the sequence is 1.
\hfill $\blacksquare$

\paragraph{The Limit of the Sequence $a_n = (1+1/n)^n$}
A fundamental limit in calculus is the limit of the sequence $a_n = (1+1/n)^n$. This limit exists, and its value is the base of the natural logarithm, the number $e$.
\[ \lim_{n \to \infty} \left(1 + \frac{1}{n}\right)^n = e \]
The proof of the existence of this limit is more involved than the previous examples. It relies on the Monotone Convergence Theorem, which states that a sequence that is both monotonic (either non-decreasing or non-increasing) and bounded must converge.

We will prove that the sequence $a_n$ is monotonically increasing and bounded above.

\textbf{1. Monotonically Increasing ($a_n \le a_{n+1}$)}
Using the binomial theorem, we can expand the term $a_n$:
\begin{align*}
a_n = \left(1 + \frac{1}{n}\right)^n &= \sum_{k=0}^{n} \binom{n}{k} \left(\frac{1}{n}\right)^k \\
&= 1 + \binom{n}{1}\frac{1}{n} + \binom{n}{2}\frac{1}{n^2} + \dots + \binom{n}{n}\frac{1}{n^n} \\
&= 1 + 1 + \frac{n(n-1)}{2!}\frac{1}{n^2} + \frac{n(n-1)(n-2)}{3!}\frac{1}{n^3} + \dots \\
&= 1 + 1 + \frac{1}{2!}\left(1-\frac{1}{n}\right) + \frac{1}{3!}\left(1-\frac{1}{n}\right)\left(1-\frac{2}{n}\right) + \dots
\end{align*}
Similarly, for $a_{n+1}$:
\[
a_{n+1} = 1 + 1 + \frac{1}{2!}\left(1-\frac{1}{n+1}\right) + \frac{1}{3!}\left(1-\frac{1}{n+1}\right)\left(1-\frac{2}{n+1}\right) + \dots
\]
By comparing the terms of the expansions for $a_n$ and $a_{n+1}$, we can see that for each $k \ge 2$, the term
\[ \frac{1}{k!}\left(1-\frac{1}{n}\right)\left(1-\frac{2}{n}\right)\dots\left(1-\frac{k-1}{n}\right) \]
is less than the corresponding term for $a_{n+1}$:
\[ \frac{1}{k!}\left(1-\frac{1}{n+1}\right)\left(1-\frac{2}{n+1}\right)\dots\left(1-\frac{k-1}{n+1}\right) \]
because for each factor, $\left(1-\frac{j}{n}\right) < \left(1-\frac{j}{n+1}\right)$ for $j=1, 2, \dots, k-1$.
Furthermore, the expansion of $a_{n+1}$ has one additional positive term. Thus, we can conclude that $a_n < a_{n+1}$ for all $n \ge 1$. The sequence is strictly increasing.

\textbf{2. Bounded Above}
From the binomial expansion of $a_n$, we have:
\[
a_n = 1 + 1 + \frac{1}{2!}\left(1-\frac{1}{n}\right) + \frac{1}{3!}\left(1-\frac{1}{n}\right)\left(1-\frac{2}{n}\right) + \dots
\]
Since each term $\left(1-\frac{j}{n}\right)$ is less than 1, we can establish an upper bound:
\[ a_n < 1 + 1 + \frac{1}{2!} + \frac{1}{3!} + \dots + \frac{1}{n!} \]
Now, we can use the inequality $k! \ge 2^{k-1}$ for $k \ge 2$. This gives us $\frac{1}{k!} \le \frac{1}{2^{k-1}}$.
\begin{align*}
a_n &< 1 + \left(1 + \frac{1}{2^1} + \frac{1}{2^2} + \dots + \frac{1}{2^{n-1}}\right)
\end{align*}
The expression in the parentheses is a finite geometric series with first term 1 and common ratio $1/2$. The sum is:
\[ S_n = \frac{1 - (1/2)^n}{1 - 1/2} = 2\left(1 - \frac{1}{2^n}\right) < 2 \]
Therefore, for all $n$, we have:
\[ a_n < 1 + 2 = 3 \]
The sequence is bounded above by 3.

\textbf{Conclusion}
Since the sequence $(a_n)_{n=1}^\infty$ is monotonically increasing and bounded above, it is convergent. The limit is defined as Euler's number, $e$.
\hfill $\blacksquare$

\subsection*{Limit of a Function}

The limit of a function at a point can be defined in two equivalent ways: the Cauchy definition (using $\varepsilon$ and $\delta$) and the Heine definition (using sequences).

\paragraph{Cauchy's Definition (\(\varepsilon\)-\(\delta\))}
Let $f$ be a real-valued function defined on a subset $D \subseteq \mathbb{R}$, and let $c$ be a limit point of $D$. A number $L \in \mathbb{R}$ is the limit of the function $f$ as $x$ approaches $c$ if for every $\varepsilon > 0$, there exists a $\delta > 0$ such that for all $x \in D$, if $0 < |x - c| < \delta$, then $|f(x) - L| < \varepsilon$.

We write this as:
\[ \lim_{x \to c} f(x) = L \]
which is equivalent to:
\[
\forall_{\varepsilon > 0} \quad \exists_{\delta > 0} \quad \forall_{x \in D} \quad (0 < |x - c| < \delta \implies |f(x) - L| < \varepsilon)
\]
The condition $0 < |x - c|$ means that we do not consider the value of $f(c)$ itself when determining the limit.

\paragraph{Heine's Definition (Sequential)}
Let $f$ be a real-valued function defined on a subset $D \subseteq \mathbb{R}$, and let $c$ be a limit point of $D$. A number $L \in \mathbb{R}$ is the limit of the function $f$ as $x$ approaches $c$ if for every sequence $(x_n)_{n=1}^{\infty}$ of elements in $D \setminus \{c\}$ that converges to $c$, the sequence of function values $(f(x_n))_{n=1}^{\infty}$ converges to $L$.

Formally:
\[ \lim_{x \to c} f(x) = L \]
is equivalent to:
\[
\forall_{(x_n) \subseteq D \setminus \{c\}} \quad \left( \left( \lim_{n \to \infty} x_n = c \right) \implies \left( \lim_{n \to \infty} f(x_n) = L \right) \right)
\]

\paragraph{Equivalence}
The Cauchy and Heine definitions of the limit of a function are equivalent. This is a fundamental theorem in real analysis.

\subsection*{Algebra of Limits}

In this section, we present the main theorems regarding arithmetic operations on limits. We will focus on sequences, but analogous theorems hold for functions.

\paragraph{Theorem (Sum of Limits)}
Let $(a_n)$ and $(b_n)$ be two sequences of real numbers such that $\lim_{n \to \infty} a_n = A$ and $\lim_{n \to \infty} b_n = B$. Then the limit of the sum of these sequences exists and is equal to the sum of their limits:
\[
\lim_{n \to \infty} (a_n + b_n) = A + B
\]

\paragraph{Proof}
By the definition of a limit, we have:
\[
\forall_{\varepsilon_A > 0}\exists_{N_A \in \mathbb{N}} : \forall_{n > N_A} \quad |a_n - A| < \varepsilon_A
\]
\[
\forall_{\varepsilon_B > 0}\exists_{N_B \in \mathbb{N}} : \forall_{n > N_B} \quad |b_n - B| < \varepsilon_B
\]
We want to show that for any $\varepsilon > 0$, there exists an $N$ such that for all $n > N$, $|(a_n + b_n) - (A + B)| < \varepsilon$.

Let $\varepsilon > 0$ be given. Let's choose $\varepsilon_A = \varepsilon/2$ and $\varepsilon_B = \varepsilon/2$. From the definitions of the limits of $(a_n)$ and $(b_n)$, there exist $N_A$ and $N_B$ corresponding to these choices.

Let $N = \max(N_A, N_B)$. Then for any $n > N$, both inequalities $|a_n - A| < \varepsilon/2$ and $|b_n - B| < \varepsilon/2$ hold.

Using the triangle inequality, we can write:
\begin{align*}
|(a_n + b_n) - (A + B)| &= |(a_n - A) + (b_n - B)| \\
&\le |a_n - A| + |b_n - B| \\
&< \frac{\varepsilon}{2} + \frac{\varepsilon}{2} \\
&= \varepsilon
\end{align*} 
Thus, we have shown that for any $\varepsilon > 0$, there exists an $N$ such that for all $n > N$, $|(a_n + b_n) - (A + B)| < \varepsilon$. This completes the proof.
\hfill $\blacksquare$

\paragraph{Theorem (Product of Limits)}
Let $(a_n)$ and $(b_n)$ be two sequences such that $\lim_{n \to \infty} a_n = A$ and $\lim_{n \to \infty} b_n = B$. Then:
\[
\lim_{n \to \infty} (a_n \cdot b_n) = A \cdot B
\]

\paragraph{Proof}
Since the sequence $(a_n)$ is convergent, it is bounded. This means there exists a real number $M > 0$ such that $|a_n| \le M$ for all $n \in \mathbb{N}$.

Let $\varepsilon > 0$ be given. From the definitions of the limits of $(a_n)$ and $(b_n)$:
For $\frac{\varepsilon}{2M} > 0$, there exists an $N_B \in \mathbb{N}$ such that for all $n > N_B$:
\[ |b_n - B| < \frac{\varepsilon}{2M} \]
For $\frac{\varepsilon}{2(|B| + 1)} > 0$ (we use $|B|+1$ to avoid division by zero if $B=0$), there exists an $N_A \in \mathbb{N}$ such that for all $n > N_A$:
\[ |a_n - A| < \frac{\varepsilon}{2(|B| + 1)} \]
Let $N = \max(N_A, N_B)$. For any $n > N$, we can estimate the difference $|a_n b_n - AB|$:
\begin{align*}
|a_n b_n - AB| &= |a_n b_n - a_n B + a_n B - AB| \\
&= |a_n(b_n - B) + B(a_n - A)| \\
&\le |a_n(b_n - B)| + |B(a_n - A)| \quad \text{(by triangle inequality)} \\
&= |a_n| |b_n - B| + |B| |a_n - A| \\
&\le M |b_n - B| + |B| |a_n - A| \\
&< M \left( \frac{\varepsilon}{2M} \right) + |B| \left( \frac{\varepsilon}{2(|B| + 1)} \right) \\
&= \frac{\varepsilon}{2} + \frac{|B|}{|B|+1} \cdot \frac{\varepsilon}{2} \\
&< \frac{\varepsilon}{2} + \frac{\varepsilon}{2} = \varepsilon
\end{align*}
This shows that $\lim_{n \to \infty} (a_n \cdot b_n) = AB$.
\hfill $\blacksquare$

\paragraph{Theorem (Quotient of Limits)}
Let $(a_n)$ and $(b_n)$ be two sequences such that $\lim_{n \to \infty} a_n = A$ and $\lim_{n \to \infty} b_n = B$. If $B \neq 0$ and $b_n \neq 0$ for all $n$, then:
\[
\lim_{n \to \infty} \frac{a_n}{b_n} = \frac{A}{B}
\]